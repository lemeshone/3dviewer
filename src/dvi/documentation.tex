\documentclass{article}
\usepackage{enumitem}
\usepackage{hyperref}
\usepackage[russian]{babel} 
\usepackage{fontenc}
\usepackage[T2A]{fontenc} 
\usepackage{times} 
\usepackage{helvet} 
\usepackage{courier} 

\usepackage{geometry}
\geometry{top=0cm, left=1cm, right=1cm, bottom=1cm, includeheadfoot}

\usepackage{titlesec}
\titlespacing*{\section}{0pt}{0pt}{0pt}
\titlespacing*{\subsection}{0pt}{0pt}{0pt}
\let\oldnewpage\newpage
\renewcommand{\newpage}{\oldnewpage}

\begin{document}

\title{Документация для проекта 3DViewer v2.0}
\date{}
\author{}
\maketitle

\section{Структура проекта}

\subsection{Модель (Model) (\texttt{src/model/})}

\begin{itemize}[label=--]
    \item Классы для представления 3D-модели: вершины, грани, матрицы преобразований
    \item Парсер файлов формата OBJ
    \item Реализация аффинных преобразований (перемещение, поворот, масштабирование)
\end{itemize}

\subsection{Контроллер (Controller) (\texttt{src/controller/})}
\begin{itemize}[label=--]
   \item Тонкие контроллеры, связывающие модель и представление
   \item Обработка пользовательского ввода и передача команд модели
   \item Реализация паттернов проектирования (Фасад, Команда, Стратегия)
\end{itemize}

\subsection{Представление (View) (\texttt{src/view/})}
\begin{itemize}[label=--]
    \item Графический интерфейс пользователя на базе Qt
    \item Визуализация 3D-модели с помощью OpenGL
    \item Элементы управления для преобразований модели
    \item Отображение информации о модели
\end{itemize}

\section{Сборка проекта}
Проект использует систему сборки \texttt{make} с Makefile, включающим следующие цели:

\begin{itemize}
    \item \texttt{all}: Сборка проекта.
    \item \texttt{install}: Установка программы в систему.
    \item \texttt{uninstall}: Удаление программы из системы.
    \item \texttt{clean}: Очистка временных файлов и папок.
    \item \texttt{dvi}: Создание файла DVI.
    \item \texttt{dist}: Создание архива, содержащего необходимые файлы для сборки и использования программы.
    \item \texttt{clang check}: Проверка на необходимость форматирования кода.
    \item \texttt{clang format}: Форматирование кода.

    \item \texttt{rebuild}: Удаление программы и новая установка.
\end{itemize}

\section{Инструкции по установке и запуску}

\begin{enumerate}
    \item \textbf{Установка зависимостей:}
        \begin{itemize}
            \item Установите компилятор g++ с поддержкой C++20 для сборки проекта.
            \item Установите библиотеку Qt6.
	  \item Установите библиотеки для тестирования: gtest.
        \end{itemize}
    \item \textbf{Установка:}
        \begin{itemize}
            \item Выполните \texttt{make install} для установки программы.
        \end{itemize}
    \item \textbf{Запуск:}
        \begin{itemize}
            \item Выполните \texttt{make run} для запуска программы.
        \end{itemize}
\end{enumerate}

\section{Использование программы 3DViewer}

\begin{enumerate}
	\item \textbf{Загрузка модели:}
	   \begin{itemize}
		\item Нажмите кнопку "Open File" для выбора файла .obj
		\item Программа поддерживает списки вершин и поверхностей
		\item Название файла отображается в интерфейсе
	\end{itemize}
\item \textbf{Просмотр информации о модели:}
    \begin{itemize}
        \item Интерфейс отображает количество вершин и рёбер
        \item Поддерживаются модели от 100 до 1,000,000 вершин
    \end{itemize}

\item \textbf{Преобразования модели:}
    \begin{itemize}
        \item \textbf{Перемещение}: Введите значения для осей X, Y, Z
        \item \textbf{Поворот}: Укажите углы поворота вокруг осей
        \item \textbf{Масштабирование}: Задайте коэффициент масштабирования
    \end{itemize}
\end{enumerate}

\section{Тестирование}

Проект включает в себя unit-тесты с использованием библиотеки \texttt{gtest}.

\end{document}
